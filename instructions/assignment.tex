% causal broadcast chat
% extension tasks?
%
% whether it works
% whether it's fault tolerant
% not client server
% quality of code
% quality of report
%


\documentclass[a4paper]{article}

% Use utf-8 encoding for foreign characters
\usepackage[utf8]{inputenc}

% Setup for fullpage use
% \usepackage{fullpage}

% Uncomment some of the following if you use the features
%
% Running Headers and footers
%\usepackage{fancyhdr}

% Multipart figures
%\usepackage{subfigure}

% More symbols
\usepackage{amsmath}
\usepackage{amssymb}
%\usepackage{latexsym}
\newcommand{\union}{\cup}

% Surround parts of graphics with box
% \usepackage{boxedminipage}

% Package for including code in the document
\usepackage{listings}

\usepackage[ruled,vlined]{algorithm2e}

\usepackage{hyperref}


% This is now the recommended way for checking for PDFLaTeX:
\usepackage{ifpdf}

%\newif\ifpdf
%\ifx\pdfoutput\undefined
%\pdffalse % we are not running PDFLaTeX
%\else
%\pdfoutput=1 % we are running PDFLaTeX
%\pdftrue
%\fi

\ifpdf
\usepackage[pdftex]{graphicx}
\else
\usepackage{graphicx}
\fi
\title{LSINF2345 Assignment 2013: Dancing robots}
\author{Nicholas Rutherford}

\date{April 2013}


\begin{document}
\lstset{language=erlang}

\ifpdf
\DeclareGraphicsExtensions{.pdf, .jpg, .tif}
\else
\DeclareGraphicsExtensions{.eps, .jpg}
\fi

\maketitle


\section{Introduction}

Shamshung's robotics team are deploying a new publicity campaign this year:
georeplicated dancing robots. The robots will dance in 4 different venues
simultaneously, connected by video, making up their own dance routine live.
They have to be careful that nothing goes wrong, as a major rival,
Babble, will be watching carefully for mistakes to discuss over lunch with
their customers.

Unfortunately for Shamsung, their robotic dancers turned out to be rather
temperamental, and are prone to spectacular -- often explosive -- failure
during performances. This is unfortunate because the team's software engineers
decided one robot would lead, deciding on a dance routine and telling the
other robots which steps to perform. This is not predetermined: the robots
decide the next step just before taking it. They had assumed their robots
would live forever.

In order to keep the show moving we need to substitute this
central-point-of-failure with a suitable alternative. In this assignment you
will design two (or more) solutions which keep the robots dancing using
distributed algorithms from the course. You should explain your approach and
provide a prototype implementation using the provided Erlang stack and
dancing robot simulator.

Management's specification for the dancers:

  - the robots will start from the same pose
  - the robots will perform the same dance routine: the same steps in the same order,
    as if it had been written down before the show and given to them
  - no robot can miss a step: if one robot takes it they all take its
  - all robots must finish in the same pose and the same place on the stage

  - (after the show the producer wants a print-out of the dance routine (to sell at auction))
  - (no two robots can be out-of-sync by more than 10 seconds.)
  - (create an application layer which selects a robot from the group to
      display on the local TV screen. It should announce a robot every 10 seconds,
      and should eventually stop displaying crashed robots).





\section{Dealing with delayed and lost messages} % (fold)
\label{sec:dealing_with_unreliable_links}

First of all we will assume no robot fails, but will introduce losses and
delays in the point-to-point links. To see the problem, execute the program
with the fll component's ``reliable'' state field set to false.

You will see that, unlike when it was set to true, the robots soon move out
of sequence. In this first exercise you should fix this by guaranteeing
delivery and ordering the messages.



% section dealing_with_unreliable_links (end)







\section{Assumptions} % (fold)
\label{sec:assumptions}

\subsection{Crash-stop failures} % (fold)
\label{sub:crash_stop_failures}

Once a robot has exploded it will not recover. There are no Byzantine robots
in the group (we hope).

% subsection crash_stop_failures (end)

\subsection{Broadcast, failure detection and lost messages} % (fold)
\label{subsec:lost_messages}

Your design may require the use of a failure detector. This may become
problematic in broadcast algorithms where correct robots are falsely suspected
if you use \emph{$\Diamond$P} or set the timeout of \emph{P} too low.
Reasoning about \emph{$\Diamond$P} is left as an extension task. Start with
\emph{P}, and consider the following for selecting its time bound.

In order to select a time bound, determine how long is a reasonable time for
a perfectly delivered message to be sent (heartbeat request) and returned
(heartbeat response) such that some number of message losses is allowed.

1 in 10 messages are lost by the \emph{fll} connecting the robots. Reason
probabilistically about these message losses, determine how many you are
prepared to say will never in-practice be lost in a row, and take that as the
maximum retransmissions required by your perfect link abstraction to achieve
delivery. Include this reasoning in your report. You may find the performance
subheadings of \cite{cachin2011}'s algorithm descriptions useful in
determining the time taken.

% If using the perfect failure detector, set its timeout so that 2 messages can
% be lost without failure being detected. The maximum round-trip time is 5
% seconds, but you must also consider the retransmission delay of \emph{sl}.

% subsection lost_messages (end)



% section assumptions (end)





\section{Optimising the perfect-link abstraction} % (fold)
\label{sub:optimising_pl}

The \emph{pl}, \emph{sl}, \emph{fll} messaging stack is theoretically correct,
but expensive in practice. Can you modify it so that fewer messages are sent,
without risking that messages are not delivered?

Can you think of a way to safely garbage-collect the stored and retransmitted
messages?



% subsubsection halting_retransmission_on_algorithm_termination (end)

\subsection{Other ideas} % (fold)
\label{ssub:other_ideas}

How else might you optimise \emph{pl}, \emph{sl}, \emph{fll}? Check the performance clauses of
sections 2.4.3-4 in the course book, about \emph{sl} and \emph{pl}, for ideas.

Consider the consequences of node failures and the various failure models
with the algorithms and your changes.

% subsubsection other_ideas (end)

% subsection optimising_pl (end)








\section{Practical details} % (fold)
\label{sec:practical_details}


\subsection{Installing Erlang with WX} % (fold)
\label{sub:installing_erlang_and_wx}

Be sure to install (or build) Erlang with wxWidgets for the graphical
component of the assignment.

Mac users (10.8) can install 32-bit Erlang from
\url{https://www.erlang-solutions.com/downloads/download-erlang-otp}. This
company's builds may also be useful for other platforms.



% subsection installing_erlang_and_wx (end)



\subsection{Where to find help} % (fold)
\label{sub:where_to_find_help}

In order to find documentation for Erlang functions, try Google, the reference
manual\footnote{\url{http://www.erlang.org/doc/apps/stdlib/index.html}}, and
possibly \url{http://erldocs.com}.
To better understand the algorithms, refer to the course slides or textbook
\cite{cachin2011}.

You can email me questions about your design and problems with Erlang. Please
email to arrange in-person meetings.

It's ok to talk to the other students about how to solve the problems, but
don't copy their code.

% subsection where_to_find_help (end)



% section practical_details (end)







\begin{thebibliography}{9}

\bibitem{cachin2011}
  C. Cachin, R. Guerraoui, and L. Rodrigues,
  \emph{Introduction to Reliable and Secure Distributed Programming}.
  Springer-Verlag, Berlin
  2nd Edition,
  2011.

\end{thebibliography}

\end{document}


